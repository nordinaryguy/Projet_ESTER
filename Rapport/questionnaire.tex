\section{Questionnaires}

\subsection{Création de questionnaires}
La problématique consistait à proposer un outil de création de questionnaires facile à utiliser. Cet outil était un besoin fondamental dans le projet qui nous a été donnés. La solution que nous avons adopté pour répondre au besoin formulé par le client, était de créer un générateur de questionnaires. L’utilisateur, à l'aide de celui-ci, peut créer des formulaires HTML en glissé-déposé. Étant donné que les utilisateurs de cet outil sont des non-informaticiens, on a adapté cet outil aux besoins de ces derniers pour le rendre plus facile aussi bien à l’utilisation qu'à la compréhension de son fonctionnement.

\begin{figure}[H]
    \begin{center}
	\includegraphics[scale=0.8]{img/questionnaire/generateur}
    \end{center}
    \caption{Ajout d'une question dans le générateur de questionnaires}
\end{figure}



Cette interface permet à l'utilisateur de créer un questionnaire et paramétrer ses questions en glissant les questions de la partie droite vers la partie gauche et en cliquant sur la question déposée, un nouveau formulaire apparaît avec un nombre de zones de texte différent selon le type de question. Ce formulaire permet de modifier ou remodifier la question.


\begin{figure}[H]
    \begin{center}
	\includegraphics[scale=0.8]{img/questionnaire/modification}
    \end{center}
    \caption{Modification d'une question}
\end{figure}
 
 
Pour supprimer une question, il suffit de glisser la question hors du cadre. Cela permet de la retirer des autres questions.


\begin{figure}[H]
    \begin{center}
	\includegraphics[scale=0.8]{img/questionnaire/suppresion}
    \end{center}
    \caption{Suppression d'une question}
\end{figure}


Lorsqu'on souhaite repositionner les questions, nous avons fait en sorte qu'il suffise de glisser puis déposer les questions à l'endroit où nous souhaitons la placer (en restant dans le même cadre).



\begin{figure}[H]
    \begin{center}
	\includegraphics[scale=0.8]{img/questionnaire/repositionnement}
    \end{center}
    \caption{Repositionnement des questions}
\end{figure}


Pour finaliser la création d’un questionnaire, il faut que l’utilisateur saisisse le nom et l’identifiant du questionnaire puis en cliquant sur le bouton "enregistrer". Le questionnaire va être sauvegardé en deux formes différentes, la première est sous forme d’un fichier HTML (figure 3.9) et la deuxième est dans la base de données qu’on peut consulter à partir de la liste des questionnaires (figure 3.10).


\begin{figure}[H]
    \begin{center}
	\includegraphics[scale=0.7]{img/questionnaire/enregistrement}
    \end{center}
    \caption{Saisie de données pour finaliser la sauvegarde}
\end{figure}


\begin{figure}[H]
    \begin{center}
	\includegraphics[scale=0.7]{img/questionnaire/fichier}
    \end{center}
    \caption{Sauvegarde sous forme d'un formulaire HTML}
\end{figure}

\begin{figure}[H]
    \begin{center}
	\includegraphics[scale=0.7]{img/questionnaire/affichage}
    \end{center}
    \caption{Affichage du questionnaire enregistré dans la base de données}
\end{figure}

\subsection{Questionnaire Eval\_Risk\_TMS}

Dans le cahier des charges, il était spécifié qu'un questionnaire devait impérativement être créé : Eval\_Risk\_TMS est un questionnaire relatif à la médecine du travail. Il fut le premier créé, sous forme d'un script HTML. Il a été enregistré dans un fichier HTML ainsi que dans la base de données en tant que String.
Il est ensuite appelé dans la partie d'affichage des utilisateurs où les salariés peuvent y répondre ou encore le personnel soignant peut regarder les pages, une part une, dont Eval\_Risk\_TMS.

\begin{figure}[H]
    \begin{center}
	\includegraphics[scale=0.4]{img/Sante}
    \end{center}
    \caption{Logo de Santé publique France}
\end{figure}