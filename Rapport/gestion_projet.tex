\chapter{Gestion du projet}

\section{Présentation du projet}

En lien avec l'équipe ESTER, nous avons reçu, pour tâche, de réaliser un site web permettant de créer des questionnaires médicaux auxquels pourraient répondre des salariés de diverses entreprises. Selon le cahier des charges et tout en respectant le secret professionnel, nous devions faire en sorte que les patients puissent répondre aux questions que les médecins auraient préparé dans des questionnaires, enregistrés dans une base de données. Une fois créé et enregistré, le corps médical devait pouvoir les attribuer en fonction des cas. Par la suite, des résultats devaient être calculé et affiché aux personnels soignants afin qu'ils puissent se rendre compte des chiffres. 

\section{Choix technologies}

\subsection{Technologie côté serveur}

Nous avons proposer d'utilisé NodeJS coté serveur, qui avais pour avantage d'être bien intégré avec MongoDB
et de nous permetre d'utilisé un seul technologies qui est le JavaScript. Cela nous aurrais économiser du 
temps de développément.

\subsubsection{JEE}

Nos chefs de projet ont décidé d'utiliser le JEE car c'est une technologie qu'ils connaisaient et car le 
JEE nous sera utilie lors du prochain semestre. JEE est la version spécialise de Java pour la création 
d'application pour les entreprises et permet de développé des service web.

\subsection{Technologie côté client}

JavaScript : Est un langage de programmation développé par Netscape en 1995 sous le nom de LiveScript. Il s'agit d'un langage de script léger, orienté objet, principalement connu comme le langage de script des pages web. \

CSS : « Cascading Style Sheets » ce qui signifie « feuille de style en cascade ». 
Il s'agit d'un langage informatique utilisé pour mettre en forme les fichiers HTML ou XML. \

Bootstrap : Correspond à une collection d'outils, développé depuis 2010, utiles pour la création de sites web. Cette collection contient des codes HTML et CSS ainsi que des extensions JavaScript.


Le projet, nous demande une certaine mise en forme (voir figure si dessous) des résultats. Dans les besoins, il avais aussi d'exporter les donnée dans diverses formats CSV, PDF, PNG et autres et d'être compatible avec les PC/Tablette/Smartphone.  

\subsubsection{Highcharts}

Highcharts est un librairie JavaScript qui permet de généré des graphiques interactifs. Les paramétrages s'effectue en JSON et offre la possibilité export dans les différents formats.

\subsection{Base de données}

Pour le projet, nous avons eu besoin d'une base de données modulaire car un des besoins était que les questionnaires pouvait évoluer création, modification ou ajout de question. Et un autre des besoins était que l'utilisateur puisse faire des sauvegarde partiel pour reprendre le questionnaires en case de problème ou si l'utilisateur veut faire une pause. 

En plus des besoins spécifique pour la sauvegarde des questionnaires et des réponses. Il y a des besoins plus génériques comme la gestion des comptes que nous verrons plus en détails dans un autre partie.

\subsubsection{MongoDB}

Nous somme partie sur du MongoDB qui fais partie de la mouvance NoSQL qui s'écarte du paradigme classique des bases relationnelles. Cela nous permet de nous affranchir d'une des contraintes des base de données SQL qui est de devoir définir un schéma prédéfini. Nous somme quand même partie d'un schéma de base pour avoir des données en partie structurée.

En offrant un plus grande flexibilité en permettant de gérer des données hétérogènes. Dans cas cela est particulièrement utile pour les questionnaires et les réponses car si un questionnaires est modifier, il faut que les anciens réponses reste en parti utilisable.

\begin{figure}[H]
    \begin{center}
    \includegraphics[height=2.0cm]{img/mongodb}
    \end{center}
    \caption{Logo de MongoDB}
\end{figure}

Ce choix du type de la base de données a été proposer par nos chefs de projet. La raison du choix de MongoDB est car il est le membre le plus populaire de la famille NoSQL.   

\section{Planification et répartition des tâches}

\subsection{Outils utilisés}


\subsection{Diagramme de Gantt}


\subsection{Répartitions des rôles}