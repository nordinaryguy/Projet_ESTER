\chapter{Gestion du projet}

\section{Présentation du projet}



\section{Choix technologies}

\subsection{Technologie côté serveur}

\subsubsection{JEE}

Nous avons proposer d'utilisé NodeJS côté serveur, car il s'intègrais particulièrement 
bien a MongoDB et propose une maniere simple d'intégrer des modules. De plus pour la 
parti formation nous aurions du simplement apprendre le javascript et ainsi éviter 
d'apprendre une autre technologie.

Mes nos chefs de projet on décider de choisir le JEE, car c'êtait une technologie 
qu'ils connaissait et cela nous permetais de découvrire la technologie qui nous sera 
utile pour le second semestre.

\subsection{Technologie côté client}


\subsubsection{Highcharts}



\subsection{Base de données}

\subsubsection{Besoin}

Pour le projet, nous avons eu besoin d'une base de données modulaire car un des besoins
était que les questionnaires pouvait évoluer création, modification ou ajout de 
question. Et un autre des besoins était que l'utilisateur puisse faire des sauvegarde 
partiel pour reprendre le questionnaires en case de problème ou si l'utilisateur veut 
faire une pause. 

En plus des besoins spécifique pour la sauvegarde des questionnaires et des réponses. 
Il y a des besoins plus génériques comme la gestion des comptes que nous verrons plus 
en détails dans un autre partie.

\subsubsection{MongoDB}

Nous somme partie sur du MongoDB qui fais partie de la mouvance NoSQL qui s'écarte du 
paradigme classique des bases relationnelles. Cela nous permet de nous affranchir 
d'une des contraintes des base de données SQL qui est de devoir définir un schéma 
prédéfini. Nous somme quand même partie d'un schéma de base pour avoir des données en 
partie structurée.

En offrant un plus grande flexibilité en permettant de gérer des données hétérogènes. 
Dans cas cela est particulièrement utile pour les questionnaires et les réponses car 
si un questionnaires est modifier, il faut que les anciens réponses reste en parti 
utilisable.

\begin{figure}[H]
    \begin{center}
    \includegraphics[height=2.0cm]{img/mongodb}
    \end{center}
    \caption{Logo de MongoDB}
\end{figure}

Ce choix du type de la base de données a été proposer par nos chefs de projet. La 
raison du choix de MongoDB est car il est le membre le plus populaire de la famille 
NoSQL.   

\section{Planification et répartition des tâches}

\subsection{Outils utilisés}


\subsection{Diagramme de Gantt}


\subsection{Répartitions des rôles}