\section{Résultat}

L'implémentation de base de Highcharts a été assez simple, mais la création des lignes de séparation (exemple "D'accord vs Pas d'accord") et de coche pour les réponses du le salarie on demandé plus de temps. 

Ce temps en plus est du au temps à la nécessité pour l'adapter a nos besoins. Car certaine n'avais pas des fonctions par défaut pour efféctuer les taches si dessus.

\subsubsection{DWR}

DWR (Direct Web Remoting) est un librairie Java qui permet de recevoir des résultats du serveur sur le principe Ajax (asynchronous JavaScript and XML) qui permet de faire des requêtes aux serveur, mais de manière simplifier.

Cette librairie nous permet de faire le line entre Highcharts et notre serveur pour récupéré les information telle que les questions et leurs valeurs (réponse et pourcentage).

\begin{figure}[H]
    \begin{center}
    \includegraphics[height=2.0cm]{img/mongodb}
    \end{center}
    \caption{Mise en forme des résultat rendus final}
\end{figure}